\documentclass{article}
% \usepackage{cite}
\usepackage{algorithm}         
\usepackage[noend]{algpseudocode} % For pseudocode inside the algorithm environment
\usepackage{amsmath}
\usepackage{graphicx}
\usepackage{subcaption} % Use subcaption instead of subfigure (more modern)


% if you need to pass options to natbib, use, e.g.:
%     \PassOptionsToPackage{numbers, compress}{natbib}
% before loading neurips_2023


% ready for submission
\usepackage[preprint]{neurips_2023}


% to compile a preprint version, e.g., for submission to arXiv, add add the
% [preprint] option:
%     \usepackage[preprint]{neurips_2023}


% to compile a camera-ready version, add the [final] option, e.g.:
%     \usepackage[final]{neurips_2023}


% to avoid loading the natbib package, add option nonatbib:
%    \usepackage[nonatbib]{neurips_2023}


\usepackage[utf8]{inputenc} % allow utf-8 input
\usepackage[T1]{fontenc}    % use 8-bit T1 fonts
\usepackage{hyperref}       % hyperlinks
\usepackage{url}            % simple URL typesetting
\usepackage{booktabs}       % professional-quality tables
\usepackage{amsfonts}       % blackboard math symbols
\usepackage{nicefrac}       % compact symbols for 1/2, etc.
\usepackage{microtype}      % microtypography
\usepackage{xcolor}         % colors


\title{Reward Hacking Mitigation}


% The \author macro works with any number of authors. There are two commands
% used to separate the names and addresses of multiple authors: \And and \AND.
%
% Using \And between authors leaves it to LaTeX to determine where to break the
% lines. Using \AND forces a line break at that point. So, if LaTeX puts 3 of 4
% authors names on the first line, and the last on the second line, try using
% \AND instead of \And before the third author name.


\author{
  % David S.~Hippocampus\thanks{Use footnote for providing further information
  %   about author (webpage, alternative address)---\emph{not} for acknowledging
  %   funding agencies.} \\
  % Department of Computer Science\\
  % Cranberry-Lemon University\\
  % Pittsburgh, PA 15213 \\
  % \texttt{hippo@cs.cranberry-lemon.edu} \\
  % \And
  Joe Gaucho\\
  UC Santa Barbara\\
  \texttt{jgaucho@ucsb.edu}
  \And
  Joe Gaucho\\
  UC Santa Barbara\\
  \texttt{jgaucho@ucsb.edu}
  % examples of more authors
  % \And
  % Coauthor \\
  % Affiliation \\
  % Address \\
  % \texttt{email} \\
  % \AND
  % Coauthor \\
  % Affiliation \\
  % Address \\
  % \texttt{email} \\
  % \And
  % Coauthor \\
  % Affiliation \\
  % Address \\
  % \texttt{email} \\
  % \And
  % Coauthor \\
  % Affiliation \\
  % Address \\
  % \texttt{email} \\
}


\begin{document}


\maketitle


\begin{abstract} 
  Reward hacking is a significant concern in the training of advanced reinforcement learning (RL) agents. In complex environments and tasks, it is challenging to specify a perfect reward function that fully captures the intended human behaviors across all edge cases. RL agents that exploit misspecifications in the reward function can exhibit undesirable and potentially harmful behaviors. In this work, we investigate the problem of reward hacking and propose a novel approach that improves the approximation of the true reward function by incorporating feedback from Vision-Language Models (VLMs) into the training loss. 
\end{abstract}

\section{Introduction}

Lorem ipsum dolor sit amet, consectetur adipiscing elit, sed do eiusmod tempor incididunt ut labore et dolore magna aliqua. Ut enim ad minim veniam, quis nostrud exercitation ullamco laboris nisi ut aliquip ex ea commodo consequat. Duis aute irure dolor in reprehenderit in voluptate velit esse cillum dolore eu fugiat nulla pariatur.

\section{Background}
Example citations \cite{amodei2016concrete} \cite{pan2022effectsrewardmisspecificationmapping} \cite{skalse2022definingcharacterizingrewardhacking}


\section*{Unlabeled Section}
Lorem

%%%%%%%%%%%%%%%%%%%%%%%%%%%%%%%%%%%%%%%%%%%%%%%%%%%%%%%%%%%%
\bibliographystyle{plain}
\bibliography{references} 

\end{document}
